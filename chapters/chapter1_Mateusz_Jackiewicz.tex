\section{Mateusz Jackiewicz}
\label{sec:mjackiew}

\begin{figure}[htbp]
    \centering
    \includegraphics[width=0.2\textwidth]{pictures/lechpremium.jpg}
    \caption{Piwo lech premium}
    \label{fig:piwo}
\end{figure}

Dodaje piwo do dokumentu (zobacz figure~\ref{fig:piwo}). Kocham moją dziewczyne i piwo.
\begin{table}[h]
\centering
\begin{tabular}{|c|c|c|c|c|c|}
\hline
*  & \textbf{1} & \textbf{2} & \textbf{3} & \textbf{4} &\textbf{5}   \\ \hline
1  &    1       &     2      &      3     &     4       &   5 \\ \hline 
2  &    2       &     4      &      6     &     8       &   10\\ \hline
3  &    3       &     6      &      9     &     12      &   15 \\ \hline
4  &    4       &     8      &      12     &     16     &   20\\ \hline
5  &    5       &     10      &      15     &     20     &   25\\ \hline
\end{tabular}
\caption{Tabliczka mnożenia}
\label{tab Z4}
\end{table}

Top 3 piwa:
\begin{enumerate}
    \item Harnaś.
    \item Tatra.
    \item EB\\\\
\end{enumerate}

Wzory z fizy
\begin{itemize}
    \item $ S = (at^2)/2 $
    \item $ F = ma $
    \item $ V = S/t $\\\\
\end{itemize}
{\Large{\bf Przepis na bigos:}\newline}
Mięso pokroić w kostkę. Cebulę pokroić w kosteczkę i zeszklić na oleju w dużym garnku. Dodać mięso i dokładnie je obsmażyć.
    
Wlać 2 szklanki gorącego bulionu lub wody z solą i pieprzem, zagotować. Następnie dodać połamane suszone grzyby, przykryć, zmniejszyć ogień i gotować przez ok. 45 minut.
    
Dodać listek laurowy, ziela angielskie, kminek, majeranek, powidła śliwkowe lub posiekane śliwki, obrane i pokrojone w kosteczkę obrane jabłko i wymieszać.
    
Dodać odciśniętą kiszoną kapustę oraz wlać szklankę wody, wymieszać. Przykryć i gotować przez ok. 15 minut.
    Kiełbasę obrać ze skóry, pokroić w kostkę i podsmażyć na patelni. Dodać do kapusty i gotować przez ok. 30 minut. Pod koniec dodać koncentrat pomidorowy.
    
Mąkę podsmażyć na suchej patelni, gdy zacznie brązowieć dodać łyżkę masła i mieszać aż masło się rozpuści.
    Trzymając patelnię na ogniu dodać stopniowo kilka łyżek kapusty cały czas mieszając. Przełożyć zawartość patelni z powrotem do garnka, wymieszać i zagotować.
    
\newpage