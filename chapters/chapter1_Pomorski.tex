\section{Jakub Pomorski}
\label{sec:mat}

Wzory na pola figur:
\begin{itemize}
    \item kwadrat $ P = a^2 $
    \item prostokąt $ P = a * b $
    \item trójkąt $ P = \frac{a * h}{2}\ $
\end{itemize}

\begin{figure}[htbp]
    \centering
    \includegraphics[scale=0.12]{pictures/image_Pomorski.jpg}
    \caption{zdjecie}
    \label{fig:zdj}
\end{figure}

\begin{table}[h]
\centering
\begin{tabular}{|c||c|c|c|c|}
\hline
+  & \textbf{0} & \textbf{1} & \textbf{2} & \textbf{3}   \\ \hline \hline
0  &    0       &     1      &      2     &     3        \\ \hline 
1  &    1       &     2      &      3     &     0        \\ \hline
2  &    2       &     3      &      0     &     1        \\ \hline
3  &    3       &     0      &      1     &     2        \\ \hline
\end{tabular}
\caption{dodawanie w grupie {\bf Z}\tiny 4}
\label{tab Z4}
\end{table}

\newpage

{\Large{\bf Przepis na naleśniki:}\newline}

{\large {\bf Składniki:}}
\begin{enumerate}
    \item 1 szkalnka mąki pszennej
    \item 2 jajka
    \item  1 szklanka mleka
    \item $\frac{3}{4}$ szklnki wody (najlepiej gazowanej
    \item szczypta soli
    \item 3 łyzki masła lub oleju roślinnego
\end{enumerate}

{\large {\bf Przygotowanie:}}
\begin{itemize}
    \item Mąkę wsypać do miski, dodać jajka, mleko, wodę i sól. Zmiksować na gładkie ciasto. Dodać roztopione masło lub olej roślinny i razem zmiksować (lub wykorzystać tłuszcz do smarowania patelni przed smażeniem każdego naleśnika).
    \item 	Naleśniki smażyć na dobrze rozgrzanej patelni z cienkim dnem np. naleśnikowej. Przewrócić na drugą stronę gdy spód naleśnika będzie już ładnie zrumieniony i ścięty.
\end{itemize}

{\Large{\bf Historia naleśników}\newline}

Obecna forma naleśników nieco różni się od tej znanej już za czasów starożytnej Grecji. Każda kultura, która przejmowała coraz bardziej popularny przepis, modyfikowała skład i smak dania. Jednak pierwsze informacje, które opisywały ciasto podobne do dzisiejszych naleśników pochodzą od greckiego komediopisarza Kratinosa. Tworzył on w Atenach w IV wieku p.n.e. i owe ciasto opisał w jednym ze swoich dzieł. Pisarz określił danie jako ciepłe i smaczne.\newline

Naleśniki były popularne nie tylko w Grecji. Odpowiednie wersje naleśników były popularne w wielu europejskich krajach. Z czasem pojawiły się konkretne cechy naleśników, na podstawie których powstało obecne danie. Różnorodność ciast naleśnikowych powoduje, że nie jest ważne dokładne zachowanie proporcji składników. Naleśniki możemy przyrządzić na słodko z serem lub dodając krem czekoladowy, albo też z owocami. Znane i równie popularne są też wersje wytrawne - z mięsem, warzywami, kapustą i grzybami, Każda kompozycja tworzy nowy smak i tym samym nowy przepis.