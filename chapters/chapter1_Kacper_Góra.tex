\section{Kacper Góra}

Zdjecie mojego awataru na SLACK. (see Figure~\ref{fig:kpi}).

\begin{figure}[htbp]
    \centering
    \includegraphics[width=0.8\textwidth]{pictures/kpi.jpg}
    \caption{Awatar}\\\\
    \label{fig:kpi}
\end{figure}

Mój tydzień.

\begin{table}[h]
\centering
\begin{tabular}{|c|c|c|c|c|c|c|c|}
\hline
godzina & \textbf{Poniedziałek} & \textbf{Wtorek} & \textbf{Środek} & \textbf{Czwartek} & \textbf{Piątek} & \textbf{Sobota} & \textbf{Niedziela} \\ \hline
8:00    &                       &                 &                 &                   &                 &                 &                    \\ \hline
10:00   &                       &                 &                 &                   &                 &                 &                    \\ \hline
12:00   &                       &                 &                 &                   &                 &                 &                    \\ \hline
14:00   &                       &                 &                 &                   &                 &                 &                    \\ \hline
16:00   &                       &                 &                 &                   &                 &                 &                    \\ \hline
18:00   &                       &                 &                 &                   &                 &                 &                    \\ \hline
20:00   &                       &                 &                 &                   &                 &                 &                    \\ \hline
22:00   &                       &                 &                 &                   &                 &                 &                    \\ \hline
\end{tabular}
\caption{Prosty planer tygodnia!!!!}
\end{table}


\newpage
OP wzory :
\begin{itemize}
    \item $ E=mc^2 $
    \item $ a^2 + b^2 = c^2 $
    \item $ a * b = b * a $ (mnożenie jest przemienne)\\\\
\end{itemize}


Fajnie jest gdy:
\begin{enumerate}
    \item Sie wyspie.
    \item Jestem najedzony.
    \item Nie widzie Pana T.\\\\
\end{enumerate}

    Wprowadzenie do formatowania tekstu:
\begin{enumerate}
    \item Pogrubienie tekstu tworzymy za pomoca (textbf{argument})
    \item Kursywa tekstu za pomoca (emph{argument})
    \item Podkreślenie tekstu za pomoca (underline{argument})\\\\
\end{enumerate}

Program \textbf{LaTeX} jest bardzo pomocny przy pisaniu prac magisterskich itp. 
Nauka kozystania z niego chwile zajuje ale jest to \emph{przednia zabawa}.\\\\

Podczas pisania teskstu najwieksza problem sprawilo mi to ze\\ \underline{nie moge wpisywac
tutaj polskich znakow}. Jednakze nie jest to takie skomplikowane do użycia.\\\\
\textbf{Jednak chyba można ????}
    
\newpage